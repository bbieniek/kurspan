\nonstopmode{}
\documentclass[a4paper]{book}
\usepackage[times,inconsolata,hyper]{Rd}
\usepackage{makeidx}
\usepackage[utf8]{inputenc} % @SET ENCODING@
% \usepackage{graphicx} % @USE GRAPHICX@
\makeindex{}
\begin{document}
\chapter*{}
\begin{center}
{\textbf{\huge Package `bbhw6'}}
\par\bigskip{\large \today}
\end{center}
\begin{description}
\raggedright{}
\inputencoding{utf8}
\item[Type]\AsIs{Package}
\item[Title]\AsIs{Runs functions for homework 6 for PAN course}
\item[Version]\AsIs{1.0}
\item[Date]\AsIs{2015-01-25}
\item[Author]\AsIs{Baltazar Bieniek}
\item[Maintainer]\AsIs{Baltazar Bieniek }\email{baltazar@bieniek.org.pl}\AsIs{}
\item[Description]\AsIs{Complex functions}
\item[License]\AsIs{GPL (>= 2)}
\item[Imports]\AsIs{Rcpp (>= 0.11.3)}
\item[LinkingTo]\AsIs{Rcpp}
\item[Suggests]\AsIs{testthat, knitr, microbenchmark}
\item[Encoding]\AsIs{UTF-8}
\item[VignetteBuilder]\AsIs{knitr}
\item[Archs]\AsIs{i386, x64}
\end{description}
\Rdcontents{\R{} topics documented:}
\inputencoding{utf8}
\HeaderA{bbhw6−package}{bbhw6}{bbhw6−package}
\aliasA{bbhw6−package}{bbhw6bpackage}{bbhw6−package}
\aliasA{bbhw6−package-package}{bbhw6bpackage}{bbhw6bpackage.Rdash.package}
%
\begin{Description}\relax
Runs the homework 6
\end{Description}
\inputencoding{utf8}
\HeaderA{mode}{mode}{mode}
%
\begin{Description}\relax
The function calculates a mode of a given integer vector
\end{Description}
%
\begin{Usage}
\begin{verbatim}
mode(x)
\end{verbatim}
\end{Usage}
%
\begin{Arguments}
\begin{ldescription}
\item[\code{x}] IntegerVector
\end{ldescription}
\end{Arguments}
%
\begin{Value}
Integer value - mode of a vector
\end{Value}
\inputencoding{utf8}
\HeaderA{perms}{perms}{perms}
%
\begin{Description}\relax
The function generates all possible combinations of a given vector
\end{Description}
%
\begin{Usage}
\begin{verbatim}
perms(x)
\end{verbatim}
\end{Usage}
%
\begin{Arguments}
\begin{ldescription}
\item[\code{x}] input vector
\end{ldescription}
\end{Arguments}
%
\begin{Value}
Matrix - the matrix that represents all possible permutations
\end{Value}
\inputencoding{utf8}
\HeaderA{simplify2array}{simplify2array}{simplify2array}
%
\begin{Description}\relax
The function simplifies a list of numeric vectors into a numeric vector
\end{Description}
%
\begin{Usage}
\begin{verbatim}
simplify2array(x)
\end{verbatim}
\end{Usage}
%
\begin{Arguments}
\begin{ldescription}
\item[\code{x}] List - expecting a list of numeric vectors
\end{ldescription}
\end{Arguments}
%
\begin{Value}
NumericVector - a list of numeric vectors
\end{Value}
\printindex{}
\end{document}
